\documentclass[]{article}
\usepackage[utf8]{inputenc}
\usepackage{hyperref}
\usepackage[spanish]{babel}
\usepackage{listings}
\usepackage{xcolor} %para el texto con colores

\newcommand{\code}[1]{{\lstinline[basicstyle=\ttfamily,mathescape]!#1!}}
\newcommand{\toolname}{\emph{Tutoriales Interactivos}}

\lstdefinelanguage{yaml}{
	%backgroundcolor=\color{white},  % choose the background color; you must add \usepackage{color} or \usepackage{xcolor}
	basicstyle=\ttfamily,      % the size of the fonts that are used for the code
	breakatwhitespace=false,         % sets if automatic breaks should only happen at whitespace
	breaklines=true,                 % sets automatic line breaking
	captionpos=none,                 % sets the caption-position to bottom
	commentstyle=\sffamily\color{gray},           % comment style
	%deletekeywords={...},           % if you want to delete keywords from the given language
	escapeinside={(*}{*)},           % if you want to add LaTeX within your code
	extendedchars=true,              % lets you use non-ASCII characters; for 8-bits encodings only, does not work with UTF-8
	frame=tb,	                   % adds a frame around the code
	keepspaces=true,                 % keeps spaces in text, useful for keeping indentation of code (possibly needs columns=flexible)
	columns=fullflexible,
	%keywordstyle=\color{blue},      % keyword style
	%language=C++,                 % the language of the code
	numbers=left,                    % where to put the line-numbers; possible values are (none, left, right)
	numbersep=5pt,                   % how far the line-numbers are from the code
	numberstyle=\tiny, % the style that is used for the line-numbers
	%rulecolor=\color{black},         % if not set, the frame-color may be changed on line-breaks within not-black text (e.g. comments (green here))
	showspaces=false,                % show spaces everywhere adding particular underscores; it overrides 'showstringspaces'
	showstringspaces=false,          % underline spaces within strings only
	showtabs=false,                  % show tabs within strings adding particular underscores
	stepnumber=1,                    % the step between two line-numbers. If it's 1, each line will be numbered
	%stringstyle=\color{mymauve},     % string literal style
	tabsize=2,	                   % sets default tabsize to 2 spaces
	title=\lstname,                   % show the filename of files included with \lstinputlisting; also try caption instead of title	
	mathescape=true,
	comment=[l]{\#}
}

% Title Page
\title{\toolname{} -- Creación de lecciones}
\author{Enrique Martín $<$\url{emartinm@ucm.es}$>$ \\ \emph{Revisor:} Nombre $<$\url{correo@ucm.es}$>$\\}


\begin{document}
\maketitle

\begin{abstract}
La herramienta \toolname{} organiza su contenido en distintas carpetas dentro del \emph{directorio de temas}. Estas carpetas corresponden a cada uno de los lenguajes de programación soportados, y contienen un fichero YAML por cada tema. A su vez, los temas están definidos como una secuencia de lecciones, que contiene distintos elementos. En este documento se explicará la organización de carpetas y el formato concreto de los ficheros de temas.
\end{abstract}


\section{Organización de carpetas}
Todo el contenido de la aplicación se almacena en la \emph{carpeta de temas}, cuya localización debe establecer el usuario desde la ventana de configuración. Esta carpeta contendrá un directorio por cada lenguaje para el que se quiera proporcionar temas, aunque pueden existir varios directorios para versiones alternativas del mismo lenguaje de programación (por ejemplo \emph{Python 2.x} y \emph{Python 3.x}, o \emph{OpenJDK 6} y \emph{Oracle Java 8}). Para determinar cuál es el lenguaje de programación asociado a un directorio se inspecciona su nombre, comprobando si contiene alguna subcadena. Actualmente se soportan 4 lenguajes de programación, y la comprobación de nombre se realiza en el siguiente orden:

\begin{enumerate}
  \item \textbf{Python}: si el nombre del directorio contiene la subcadena \textbf{<<python>>}\footnote{En todos los casos, no importa si la subcadena aparece en mayúsculas o minúsculas en el nombre del directorio.}. Para cada uno de estos directorios, la ventana de configuración mostrará una entrada para establecer la ruta del intérprete.
  \item \textbf{Java}: si contienen la subcadena \textbf{<<java>>} en su nombre. Para cada uno de estos directorios, la ventana de configuración mostrará dos entradas: una para establecer la ruta del compilador \emph{javac} y otra para establecer la ruta del entorno de ejecución \emph{java}.
  \item \textbf{C++}: si contienen la subcadena \textbf{<<c++>>}. Para cada uno de estos directorios, la ventana de configuración mostrará una entrada para establecer la ruta del compilador \emph{GNU C++} (\code{g++}) o la del fichero de entorno de \emph{Visual Studio} (\code{vcvars*.bat}). 
  \item \textbf{C\#}: si el nombre contiene la subcadena \emph{<<c sharp>>}. Para estos directorios aparecerá una entrada en la ventana de configuración para establecer la ruta del compilador \emph{Mono} (\code{mcs}) o el fichero de entorno de \emph{Visual Studio} (\code{vcvars*.bat}).
\end{enumerate}

\section{Formato de los temas}
\begin{figure}[tbp]
\begin{lstlisting}[language=yaml,basicstyle=\ttfamily\footnotesize]
Subject: 1
Title: Titulo del tema
Intro: Breve explicaci(*ó*)n del tema
Lessons:
    - Title: Explicaciones #Primera leccion
      Elements: #Lista de elementos de la leccion
          - Elem: Text #Elemento de tipo explicacion
            Content: | #Texto de varias lineas
              Las explicaciones sirven para mostrar informaci(*ó*)n al alumno. 

              Estas esplicaciones se pueden dividir en varios p(*á*)rrafos, y con 
              Markdown se pueden incluir **negritas**, *it(*a*)licas* y `c(*ó*)digo`.
          - Elem: Text
            Content: |
                **Imagen desde internet**
                ![texto alternativo](http://URL/imagen.png)

                **Imagen desde el directorio de temas con ruta relativa**
                ![triangulo](file:///img/triangulo.jpg)
  
                **Se pueden incluir enlaces en las explicaciones:**
                [Texto del enlace](https://www.ucm.es)
          - Elem: (...)
    - Title: Preguntas de tipo test #Segunda leccion
      Elements:
          - Elem: Options #Pregunta de varias opciones
            Content: Pregunta de 3 opciones y una correcta #Texto de la pregunta
            Hint: Esto es la pista general de la pregunta
            Solution: [1] #Opciones correcta, empezando en 1
            Multiple: no #Hay varias opcinones correctas?
            Options: # Lista de opciones para mostrar
                - OPCI(*Ó*)N CORRECTA
                - Opci(*ó*)n incorrecta
                - Opci(*ó*)n incorrecta
          - Elem: (...)
    - Title: Preguntas de tipo c(*ó*)igo #Tercera leccion
      Elements:
          - Elem: Code #Pregunta de codigo
            Content: | #Texto de la pregunta, varias lineas
                Estas preguntas solicitan al usuario framentos de c(*ó*)digo que
                son insertados en un programa corrector, que es evaluado.
            Gaps: 2 #Numero de huecos a rellenar
            Prompt: ["Codigo hueco 1","Codigo hueco 2"] #Informacion de cada hueco
            Hint: Esto es la pista general. 
            File: correctores/tema0c.py #Fichero corrector con 'huecos'
          - Elem: (...)
\end{lstlisting}
\caption{Ejemplo de fichero de tema con tres lecciones y varios elementos en cada lección.\label{fig:tema}}
\end{figure}

Cada directorio que aparece en la carpeta de temas puede contener varios temas. Estos temas se definirán en ficheros con formato \code{YAML}\footnote{\url{http://www.yaml.org/spec/1.2/spec.html}} y que deben tener extensión \code{.yml}. El formato \code{YAML} es similar a \code{JSON} y sirve para representar información como texto plano organizándola como diccionarios clave-valor, listas y tipos de datos básicos (booleanos, números, cadenas, etc.).

Un ejemplo de fichero de tema se puede observar en la Figura~\ref{fig:tema}, donde se han añadido comentarios de línea (que comienzan con \code{\#}) para aclarar los distintos apartados. El formato concreto es un diccionario YAML con las siguientes claves:
\begin{itemize}
\item \code{Subject}: \textbf{número} de tema. Sirve para ordenar los distintos temas de un lenguaje a la hora de mostrarlos.
\item \code{Title}: \textbf{cadena de texto} con el título del tema. 
\item \code{Intro}: \textbf{cadena de texto} que explica brevemente el objetivo del tema. Este texto puede incluir código \emph{Markdown}\footnote{\url{https://github.com/adam-p/markdown-here/wiki/Markdown-Cheatsheet}} o incluso código HMTL directamente, como veremos en la Sección~\ref{sec:explicaciones}.
\item \code{Lessons}: lista de \textbf{lecciones} dentro del tema.
\end{itemize}

Cada una de las lecciones del tema se representa a su vez como diccionarios con dos claves:
\begin{itemize}
\item \code{Title}: \textbf{cadena de texto} con el título de la lección.
\item \code{Elements}: lista de \textbf{elementos} que conforman la lección. Existen 3 tipos de elementos: \textbf{explicaciones}, \textbf{preguntas de varias opciones} y \textbf{preguntas de código}. Los distintos elementos se tratarán con detalle en las siguientes secciones.
\end{itemize}


\section{Explicaciones}\label{sec:explicaciones}
Los elementos de tipo \emph{explicación} sirven para mostrar texto y otros elementos gráficos al alumno. Se representan como un diccionario de dos claves:
\begin{itemize}
	 \item \code{Elem: Text}
	 \item \code{Content}: \textbf{cadena de texto} con el contenido que se quiere mostrar. Puede ser texto plano, código Markdown o código HTML. La opción preferida es código Markdown por ser el más sencillo y ser suficientemente flexible.
\end{itemize}

\subsection{Markdown}
Markdown proporciona muchas características para dar formato a un texto. La herramienta \toolname{} soporta como mínimo las siguientes:
\begin{itemize}
	\item \textbf{Negrita, itálica y código integrado en la línea}:
\begin{lstlisting}[language=yaml,numbers=none]
**negrita**, *italica*, `codigo de una linea`
\end{lstlisting}	
	\item \textbf{Bloques de código de varias lineas}:
\begin{lstlisting}[language=yaml,numbers=none]
```
def f(n):
    return n + 1
```
\end{lstlisting}
	\item \textbf{Cabeceras de sección}:
\begin{lstlisting}[language=yaml,numbers=none]
(*\# Cabecera 1*)
bla bla bla

(*\#\# Cabecera 2*)
bla bla bla

(*\#\#\# Cabecera 3*)
bla bla bla

\end{lstlisting}
	\item Listas numeradas y no numeradas:
\begin{lstlisting}[language=yaml,numbers=none]
* Elemento no numerado 1
    1. elemento numerado anidado 1
    1. elemento numerado anidado 2
    1. elemento numerado anidada 3
* Elemento no numerado 2
    * elemento no numerado anidado 1
    * elemento no numerado anidado 2 
\end{lstlisting}	
	\item \textbf{Imágenes}: pueden ser imágenes remotas o almacenadas en el directorio del tema.
\begin{lstlisting}[language=yaml,numbers=none]
**Imagen remota**
![texto alternativo](https://URL/imagen.png)

**Imagen desde el directorio de temas, con ruta relativa**
![triangulo](file:///img/triangulo.jpg)
\end{lstlisting}		
Todas las imágenes cuya ruta comience con \code{file://} serán consideradas imágenes locales cuya ruta es relativa al directorio donde reside el tema actual. Por ejemplo, si el tema actual reside en \code{/opt/temas/Python 3.x}, la imagen \code{file:///img/triangulo.jpg} se referirá al fichero situado en \code{/opt/temas/Python 3.x/img/triangulo.jpg}.
	\item \textbf{Enlaces} que se abren en el navegador por defecto:
\begin{lstlisting}[language=yaml,numbers=none]
[Texto del enlace](https://www.ucm.es)
\end{lstlisting}
	\item \textbf{Notación matemática \LaTeX}. Gracias el entorno MathJax\footnote{\url{https://www.mathjax.org/}} es posible incrustar código \LaTeX{} en el texto, y que este se encarga de mostrarlo adecuadamente. El código \LaTeX{} debe encerrarse entre los símbolos \code{@@} para fórmulas en la misma línea y \code{@@@} para fórmulas en párrafos nuevos:
\begin{lstlisting}[language=yaml,numbers=none]
When @@a \ne 0@@, there are two solutions to @@ax^2 + bx + c = 0@@ and they are
@@@x = {-b \pm \sqrt{b^2-4ac} \over 2a}.@@@	
\end{lstlisting}
	\item \textbf{Tablas}:
\begin{lstlisting}[language=yaml,numbers=none]
Cabecera|Cabecera|Cabecera
---|---|---
1|2|3
1|2|3
1|2|3
\end{lstlisting}	
\end{itemize}

\subsection{HMTL}
En aquellas situaciones donde la potencia de Markdown no es suficiente, se puede incluir código HMTL directamente. Esto puede servir para incrustar vídeos y otros elementos interactivos\footnote{La herramienta \toolname{} utiliza internamente un navegador basado en WebKit (\url{https://webkit.org/}), por lo que su capacidad para procesar HTML será algo menor al de los navegadores de escritorio usuales.}.
\begin{lstlisting}[language=yaml,numbers=none]
<iframe width="640" height="360" src="https://www.youtube.com/embed/PDpMgx7avzA" frameborder="0" allowfullscreen target="_self"></iframe>
\end{lstlisting}		

\section{Preguntas de varias opciones}
Explicación formato y funcionamiento preguntas tipo test

\section{Preguntas de código}
Explicación formato y funcionamiento preguntas de código. Marcar huecos y snippet.

\end{document}          
