\documentclass[]{article}
\usepackage[utf8]{inputenc}
\usepackage{hyperref}
\usepackage[spanish]{babel}
\usepackage{listings}
\usepackage{xcolor} %para el texto con coloresç
\usepackage{tocloft}
\usepackage{graphicx}

\usepackage[colorinlistoftodos,prependcaption,textsize=tiny]{todonotes}

\renewcommand{\cftsecleader}{\cftdotfill{\cftdotsep}} % Para que todo el índice tenga puntos
\newcommand{\code}[1]{{\lstinline[basicstyle=\ttfamily,mathescape]!#1!}}
\newcommand{\toolname}{\emph{Tutoriales Interactivos}}

\lstset{ %
	%backgroundcolor=\color{white},  % choose the background color; you must add \usepackage{color} or \usepackage{xcolor}
	basicstyle=\sffamily,      % the size of the fonts that are used for the code
	breakatwhitespace=false,         % sets if automatic breaks should only happen at whitespace
	breaklines=true,                 % sets automatic line breaking
	captionpos=none,                 % sets the caption-position to bottom
	commentstyle=\itshape\color{gray},           % comment style
	%deletekeywords={...},           % if you want to delete keywords from the given language
	escapeinside={(*}{*)},           % if you want to add LaTeX within your code
	extendedchars=true,              % lets you use non-ASCII characters; for 8-bits encodings only, does not work with UTF-8
	frame=tb,	                   % adds a frame around the code
	keepspaces=true,                 % keeps spaces in text, useful for keeping indentation of code (possibly needs columns=flexible)
	columns=fullflexible,
	%keywordstyle=\color{blue},      % keyword style
	keywordstyle=\sffamily\color{teal},
	numbers=left,                    % where to put the line-numbers; possible values are (none, left, right)
	numbersep=5pt,                   % how far the line-numbers are from the code
	numberstyle=\tiny, % the style that is used for the line-numbers
	%rulecolor=\color{black},         % if not set, the frame-color may be changed on line-breaks within not-black text (e.g. comments (green here))
	showspaces=false,                % show spaces everywhere adding particular underscores; it overrides 'showstringspaces'
	showstringspaces=false,          % underline spaces within strings only
	showtabs=false,                  % show tabs within strings adding particular underscores
	stepnumber=1,                    % the step between two line-numbers. If it's 1, each line will be numbered
	stringstyle=\color{blue},     % string literal style
	tabsize=2,	                   % sets default tabsize to 2 spaces
	title=\lstname,                   % show the filename of files included with \lstinputlisting; also try caption instead of title	
	mathescape=true
}

% Title Page
\title{\toolname{} - Manual de usuario \\ \emph{(Borrador)}}
\author{Enrique Martín Martín$^a$ (\url{emartinm@ucm.es}) \\ 
	Salvador Tamarit$^b$ (\url{stamarit@dsic.upv.es}) \\
	\emph{Revisor:} Jaime Sánchez Hernández$^a$ (\url{jaime@sip.ucm.es}) \\~\\[-.4cm]
	\normalsize{\emph{$^a$Dpto. de Sistemas Informáticos y Computación}}\\[-0.1cm]
	\normalsize{\emph{Fac. Informática, Universidad Complutense de Madrid}}\\[-0.1cm]
	%\normalsize{\emph{C/ Profesor José García Santesmases, 9. 28040 Madrid, Spain}}\\[-0.1cm]
	\normalsize{\emph{$^b$Dep. Sistemes Informàtics i Computació}}\\[-0.1cm]
	\normalsize{\emph{Universitat Politècnica de València}}\\[-0.1cm]
	%\normalsize{\emph{Camí de Vera, s/n. 46022 València}}\\[-0.1cm]
}


\begin{document}
\maketitle

\tableofcontents

\clearpage

\section{Instalación y ejecución}\todo{Enrique}

Para poder ejecutar la herramienta \toolname{} debéis tener instalada en vuestro sistema la última versión de Java (en el momento de escribir este manual es la versión \emph{Java SE 8u131)}. Para ello debéis acceder a la página \url{http://www.oracle.com/technetwork/java/javase/downloads/}, pulsar el botón \emph{JRE DOWNLOAD} y escoger la versión adecuada para vuestro sistema.

Una vez se dispone de la última versión de Java, la instalación de la herramienta \toolname{} es muy sencilla: únicamente es necesario descargar el programa junto con las lecciones y situarlo en alguna carpeta del sistema (por ejemplo el directorio personal o el escritorio). Si vais a utilizar \toolname{} es una asignatura, lo más probable es que el profesor os proporcione un fichero comprimido con todos los elementos incorporados que deberéis descomprimir y situar en alguna carpeta del sistema. En el caso de no disponer de este fichero comprimido proporcionado por el profesor, deberéis descargar la herramienta de su repositorio oficial. La manera más sencilla es descargar el fichero comprimido \url{https://github.com/emartinm/TutorialesInteractivos/archive/master.zip} y descomprimirlo en alguna carpeta del sistema\footnote{Si disponéis del sistema de control de versiones \emph{GIT} también podréis clonar el repositorio de la herramienta  usando la dirección \url{https://github.com/emartinm/TutorialesInteractivos.git}}.

Para ejecutar...

\begin{itemize}
	\item Windows: doble clic en .bat o en target/blablabla.jar, lanzar .bat desde Command Prompt
	\item Linux: doble clic en target/blablabla.jar, lanzar .sh desde Command Prompt
	\item Mac: doble clic en target/blablabla.jar, lanzar .sh desde Command Prompt
\end{itemize}

\section{Configuración de la aplicación}\todo{Enrique}
Explicación de la pantalla de configuración. Lenguajes soportados

\section{Navegación general}\todo{Tama}
Breve explicación de cómo llegar a una lección y los distintos apartados navegables

\section{Lecciones}\todo{Tama}
Navegación dentro de una lección. Distintas preguntas de una lección y qué hace cada botón

\section{Solución de problemas}\todo{Enrique}
Básicamente borrar el fichero progress.json y los directorios de Preferences API

\end{document}          
